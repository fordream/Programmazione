\documentclass[a4paper,NoNotes,GeneralMath]{stdmdoc}

\newcommand{\nin}{\not\in}

\begin{document}
	\section*{Basi Cicliche}

	\Altro{Lemma: Unione finita di sottospazi}
	Sia $\bbK$ un campo con infiniti elementi e $V$ uno spazio vettoriale sul campo $\bbK$. Supponiamo che si possa scrivere $V = W_1 \cup \ldots \cup W_n$, allora $\exists i \quad W_i = V$. \\
	
	Dimostrazione: Supponiamo la minimalità dell'unione, ovvero se $\exists j \tc W_j \subseteq W_1 \cup W_2 \cup \ldots \cup W_{j-1} \cup W_{j+1} \cup \ldots \cup W_n$ allora riconsideriamo $V = W_1 \cup \ldots \cup W_{j-1} \cup W_{j+1} \cup \ldots \cup W_n$ (cioè se un sottospazio è interamente contenuto nell'unione degli altri lo togliamo). \\
	Per la minimalità abbiamo $W_n \nsubseteq W_1 \cup \ldots \cup W_{n-1}$. \\
	Sia ora $u \nin W_n$ ($u \in V$) e $v \in W_n \setminus (W_1 \cup \ldots \cup W_{n-1})$ e definiamo $S = \{v + tu \mid t \in \bbK \}$. \\
	Siccome $u$ non è il vettore nullo ed il campo $\bbK$ è infinito, allora anche l'insieme $S$ è infinito. Inoltre, poiché $S \subseteq V = W_1 \cup W_2 \cup \ldots \cup W_n$, uno dei $W_i$ deve contenere infiniti vettori di $S$. \\
	Ma, se $W_n$ contenesse un'altro vettore di $S$ oltre a $v$, allora esisterebbe $t\in\bbK$ tale che $v+tu \in W_n$, ma allora $tu = (v+tu)-v \in W_n$ e quindi avremmo $u \in W_n$, assurdo per come avevamo scelto $u$. Quindi $W_n$ non può contenere infiniti elementi di $S$. \\
	Poi, se quale $W_i$ ($1\ge i < n$) contenesse due vettori distinti di $S$, allora esisterebbero $t_1 \neq t_2 \in \bbK$ tali che $v+t_1u, v+t_2 u \in W_i$. Ma allora $(t_2-t_1)v = t_2(v+t_1u) - t_1(v+t_2u) \in W_i$ e dovremmo avere $v \in W_i$, assurdo per come avevamo scelto $v$. \\
	Quindi per $1\ge i < n$, nessun $W_i$ può contenere infiniti elementi di $S$. Ma questa è chiaramente una contraddizione, il che ci dice che $\exists i \tc W_i = V$ (questo infatti ci impedirebbe di dire che $W_n \nsubseteq W_1 \cup \ldots \cup W_{n-1}$ oppure che $\exists u \nin W_n$)

\end{document}
