\documentclass[a4paper,NoNotes,GeneralMath]{stdmdoc}

\newcommand{\omissis}{$\left[ \ldots \right]$\text{ }}
\newcommand{\err}[1]{\color{red}{#1}}

\begin{document}
	\title{Errori nel Mezzedimi di GAAL}
	\autodate

	\section*{Typo}
	\begin{enumerate}
		\item ({\bf pag. 21}) {\it Nella proposizione 2.4.6 e nel teorema 2.4.7 c'è scritto molte volte $\bbK^n$ al posto di $\bbK^q$. Andrebbe cambiato in} \\
		Notazione: Fissiamo il campo $\bbK^{\err q}$. Denotiamo:
			$$e_1 = \left( \begin{array}{c} 1 \\ 0 \\ \vdots \\ 0 \end{array} \right) \in \bbK^{\err q}, \ldots, e_{\err q} = \left( \begin{array}{c} 0 \\ \vdots \\ 0 \\ 1 \end{array} \right) \in \bbK^{\err q}$$ \\
		Osservazione: $L_A(e_1) = A \cdot \left( \begin{array}{c} 1 \\ 0 \\ \vdots \\ 0 \end{array} \right) = A^1, \ldots, L_A(e_{\err q}) = A \cdot \left( \begin{array}{c} 0 \\ \vdots \\ 0 \\ 1 \end{array} \right) = A^q$, quindi \omissis \\
	\omissis Grazie all'osservazione precedente è chiaro che l'unica matrice di questo tipo può essere solo $$ A = (g(e_1) \quad \ldots \quad g(e_{\err q})) $$ \omissis
		\item ({\bf pag. 44}) {\it Nelle osservazioni sotto la Definizione 2.8.2, il terzo punto: } \\
		\omissis 3) Se $\cB = \{v_1, \ldots, v_n\}$ è base di $V$, allora $\left[ \right]_\cB (v_i) = e_i$, cioè $\left[ \right]_\cB$ trasforma $\cB$ nella base canonica di ${\err \bbK^n}$
		\item ({\bf pag. 67}) {\it Nella proposizione 3.2.12, alla terz'ultima riga $X+iY$ va sostituito con $X+tY$} \\
		\omissis poiché in quel caso $A$ e $B$ sarebbero simili grazie a $X+{\err t}Y \in \kM(n, \bbR)$
		\item ({\bf pag. 69}) {\it Nelle Osservazioni (nella riga appena sopra alla proposizione 3.3.3) } \\
		\omissis In generale, se $A^{n-1}X = \lambda^{n-1}X$, procedendo {\err c}ome sopra \omissis
		\item ({\bf pag. 119}) {\it Nell'ultima riga del $5 \implies 1$ la somma centrale $\sum_{i,j} x_i y_j$ va sostituita con } \\
		\omissis = {$\err \sum_{i,j} x_i y_j \delta_{ij}$} $= \phi(v,w)$
		\item ({\bf pag. 130}) {\it Nella proposizione 5.1.10 nella terzultima riga andrebbe } \\
		\omissis $\implies \rho_H\circ f$ è lineare e $\Dim {\err \Fix} (\rho_H \circ f) \ge 1$, \omissis
	\end{enumerate}

	\section*{Concetti}
	\begin{enumerate}
		\item ({\bf pag. 61}) {\it L' osservazione 2) in cima alla pagina ha una giustificazione falsa: Infatti le dimostrazioni NON sono un caso particolare dell'SD-equivalenza, pur essendo analoghe e facilmente deducibili da quelle sull'SD-equivalenza. \\ Si potrebbe semplicemente giustificare dicendo }
		Per la dimostrazione di questa affermazione si può riadattare quella dell' analogo enunciato relativo all'SD-equivalenza
		\item ({\bf pag. 80}) {\it Il corollario 3.5.8 con le ipotesi attuali è falso: Servirebbe } \\
		COROLLARIO 3.5.8: $f\in\End(V), q(t)\in\text{I}(f)$. Sia $q = q_1\cdot\ldots\cdot q_m$, con {\err{$\text{MCD }(q_i, q_j) = 1 \quad \text{ se }i\neq j$}} \omissis \\
		{\it Infatti usando $q_1(t) = (t-1)(t-2), q_2(t) = (t-2)(t-3), q_3(t) = (t-1)$ si trova un controesempio}
		\item ({\bf pag. 104}) {\it La proposizione 4.2.15 non è vera: infatti può capitare durante l'algoritmo di Lagrange di dover scambiare tra di loro due vettori della base (perché quello attuale è isotropo), operazione la cui matrice non ha determinante $1$ su tutti i minori principali. \\ Andrebbe modificato con: } \\
		PROPOSIZIONE 4.2.15: {\err Le trasformazioni di base con le mosse dell'algoritmo di Lagrange nel caso il vettore non sia isotropo conservano i determinanti dei minori principali} \\
		{\it Intendo cioè che non bisogna mai ricorrere alla seconda od alla terza mossa dell'algoritmo (ovvero cambiare i vettori o l'ordine dei vettori della base). \\ Nel caso del criterio di Jacobi siamo in questa situazione perché non si incorre mai in vettori isotropi}
		\item ({\bf pag. 137}) {\it La formula di Grassmann Affine nel punto 2 è probabilmente sbagliata (ovvero non funziona per due rette sghembe in $\bbR^3$). D'altra parte non so bene come aggiustarla, ma una formula che resiste a vari test empirici è } \\
		Se $H\cap L = \emptyset \implies {\err \Dim(H+L) = \Dim(H) + \Dim(L) - \Dim(W_H \cap W_L) +1}$
	\end{enumerate}

	\section*{Suggerimenti}
	\begin{enumerate}
		\item ({\bf pag. 127}) {\it Nella proposizione 5.1.5, il terzo punto è dimostrato un po' frettolosamente. Proporrei una dimostrazione più esplicita: } \\
		Poiché $A^2 = I$, $A$ è diagonalizzabile ed ha solo autovalori $1$ e $-1$. Inoltre $V_1(A)$ e $V_{-1}(A)$ sono ortogonali fra loro. Infatti se $x \in V_{-1}(A)$ e $y \in V_1(A)$, allora $$\scal{x}{y} = {}^txy = {}^t(-Ax)Ay = -{}^tx {}^tA Ay = -{}^txy = -\scal{x}{y} \quad \implies \quad \scal{x}{y} = 0 $$ \\ Ma per il punto $1) B\in V_{-1}(A)$ , quindi $B$ è ortogonale a $V_1(A) = \Fix(A)$
	\end{enumerate}
\end{document}
